\pagenumbering{arabic}

\chapter{Introduction}


\section{Battery Performance Testing}
In a survey of around 50,000 people across 25 countries, conducted by International Data Corporation, it was found that battery life is the most important factor when buying a smartphone, with 56\% Android users, 49\% iOS users and 53\% of Windows phone users stating the same.\cite{idcsurvey} Mobile device manufacturers and mobile operating system developers pay great attention to battery usage. While device manufacturers are adding larger batteries to their devices, Android has been given a ‘Doze mode’ to reduce battery usage when phone is idle, along with a ‘Battery Saver’ mode that disables aggressive network usage, animations, and other elements that drain battery, and iOS comes with a ‘Low Power’ mode which essentially is similar to Android battery saver mode. Technologies such as Qualcomm Quick Charge™ and OnePlus Dash Charge have also been developed to reduce the time needed to charge a phone. Keeping such developments in mind, it has become very important for app developers to develop apps with optimized battery performance, so as to use as less battery as possible by default, i.e. even with battery life enhancing aids turned off.

\section{Automated UI Testing}
When apps are developed at enterprise level, they are highly complex and include a large number of possible UI elements. Testing all the components manually will require large amounts of time and energy. UI automation testing, is similar to manual testing, but instead of having a user click through the application, and visually verify the data, we write code to perform tests and verify results. Automated UI testing allows developers to "fail faster" which is a key component of agile development. Being able to identify errors sooner, gives developers more time to correct any issues long before your release. UI Automation tests can be re run as frequently as needed to test for any regressions or failures due to some changes in the source code.

\section{Problem Definition}	
In this project, we will be working towards performing automated battery performance tests and UI tests for a given Citrix Mobile app.
\section{Objective}
The objective of this project is as follows:
\begin{enumerate}
	\item To gauge the current battery usage patterns of a given Citrix Mobile app
	\item To determine factors within the app that drain more battery than desired
	\item To add few UI test cases to the Citrix App test suite and improve existing tests
	\item To monitor tests for any failures or breakages
\end{enumerate}
\section{Scope}
Beneficiaries of the work include:
\begin{enumerate}
	\item The app development team who can utilize the results to optimize the app’s battery usage and perform more extensive UI tests due to newly added tests
	\item The app users, who will have a more stable and battery efficient app
\end{enumerate}
 